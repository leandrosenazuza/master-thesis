\definecolor{lightgray}{gray}{0.9}

\chapter{Apêndice B}
\label{chap:ModelLoaderMenu}

\section{ModelLoaderMenu.cs}

\lstset{
  backgroundcolor=\color{lightgray}, 
  basicstyle=\ttfamily\footnotesize,
  breaklines=true,                   
  frame=single,                    
  language=CSharp                   
}

\begin{lstlisting}
using UnityEditor;
using UnityEngine;
using System.Collections.Generic;
using System.Net.Http;
using System.Text;
using System.Threading.Tasks;

public class ModelLoaderMenu : EditorWindow
{
    private string imagePath = "";
    private Texture2D loadedImage;

    [MenuItem("Automação Ambiente/Processar Imagem")]
    public static void ShowWindow()
    {
        ModelLoaderMenu window = GetWindow<ModelLoaderMenu>("Processar Imagem");
        window.minSize = new Vector2(300, 300);
        window.maxSize = new Vector2(300, 300);
    }

    void OnGUI()
    {
        GUILayout.Label("Carregar e Processar Imagem", EditorStyles.boldLabel);

        if (GUILayout.Button("Selecionar Imagem"))
        {
            imagePath = EditorUtility.OpenFilePanel("Selecione uma imagem", "", "png,jpg,jpeg");
            Debug.Log("imagePath: " + imagePath);
        }

        imagePath = EditorGUILayout.TextField("Caminho da Imagem", imagePath);

        if (GUILayout.Button("Processar"))
        {
            ProcessImage();
        }
    }   

    async void ProcessImage()
    {
        if (string.IsNullOrEmpty(imagePath))
        {
            Debug.LogError("Nenhuma imagem carregada para processar.");
            return;
        }

        Vector3 basePosition = Vector3.zero;
        Vector3 spacing = new Vector3(10, 0, 0); 

        List<DetectionResult> detectionResults = await SendImageForPrediction(imagePath);

        if (detectionResults != null && detectionResults.Count >= 4)
        {
            string modelPath = "Assets/Model/FILTER_F10.fbx";
            GameObject model = AssetDatabase.LoadAssetAtPath<GameObject>(modelPath);

            if (model != null)
            {
                for (int i = 0; i < detectionResults.Count; i++)
                {
                    GameObject modelInstance = (GameObject)PrefabUtility.InstantiatePrefab(model);
                    if (modelInstance != null)
                    {
                        modelInstance.transform.position = basePosition + spacing * i;
                        modelInstance.transform.rotation = Quaternion.Euler(-90, 0, 0);
                        Debug.Log("Reator inserido na posição: " + modelInstance.transform.position);
                    }
                    else
                    {
                        Debug.LogError("Erro ao instanciar o modelo virtual.");
                    }
                }
            }
            else
            {
                Debug.LogError("Erro ao carregar o modelo virtual. Verifique o caminho.");
            }
        }
    }

    async Task<List<DetectionResult>> SendImageForPrediction(string imagePath)
    {
        string url = "http://192.168.12.13:5000/predict";
        var httpClient = new HttpClient();
        var content = new StringContent($"{{\"image_path\":\"{imagePath}\"}}", Encoding.UTF8, "application/json");

        HttpResponseMessage response = await httpClient.PostAsync(url, content);

        if (response.IsSuccessStatusCode)
        {
            string jsonResponse = await response.Content.ReadAsStringAsync();
            DetectionResultsWrapper wrapper = JsonUtility.FromJson<DetectionResultsWrapper>($"{{\"results\":{jsonResponse}}}");

            return wrapper.results;
        }
        else
        {
            Debug.LogError("Erro ao chamar o endpoint Flask: " + response.ReasonPhrase);
            return null;
        }
    }
}

[System.Serializable]
public class DetectionResult
{
    public string ClassName;
    public List<float> BoundingBox;
}

[System.Serializable]
public class DetectionResultsWrapper
{
    public List<DetectionResult> results;
}
\end{lstlisting}
