\chapter{Apêndice - Formulário de Consentimento}
\label{sec:consentFormPT}

\textbf{TERMO DE CONSENTIMENTO LIVRE E ESCLARECIDO}


\vspace{\baselineskip}
Você está sendo convidado(a) para participar da pesquisa intitulada ``Arquitetura de Telereabilitação baseada em realidade aumentada para apoiar o treinamento de usuários de cadeiras de rodas motorizadas'' , sob a responsabilidade dos pesquisadores Eduardo Lázaro Martins, Edgard Afonso Lamounier Jr, Alexandre Cardoso e Daniel Stefany Duarte Caetano.

Esta pesquisa pretende proporcionar a você condições de se reabilitar, ou seja, recuperar sua independência na realização das atividades diárias antes de sua mobilidade ser restringida. Acredita-se que por meio do ambiente de Telereabilitação aumentada desenvolvido, você possa com a intervenção e acompanhamento do terapeuta, executar tarefas personalizadas que atenda as suas necessidades, remotamente.  Com base em seus desafios diários, com segurança e o mais próximo possível da realidade.  Você e o terapeuta, serão capazes de interagir com este ambiente, cuja infraestrutura foi desenvolvida baseado em suas observações, e objetos virtuais que auxiliarão neste processo.

A fusão das técnicas de Realidade Aumentada (RA) com as características advindas da Telereabilitação, te possibilitará a visualização aumentada, instantaneamente, dos comandos de controle enviados a CRM remota.  Utilizando as técnicas de RA, o terapeuta definirá o trajeto a ser executado por você, onde os virtuais poderão ser utilizados como guias, e outros deverão ser evitados. O terapeuta ainda avaliará, tomará notas e acompanhará o seu treinamento. Tudo para que você realize diversas manobras para desenvolverão suas habilidades de controle da CRM.  Você será instruído, sobre como utilizar e solicitar o treinamento no sistema, e por meio de um joystick adaptado, conectado ao computador,  enviará os comandos para controlar a CRM a distância sem ser necessário deslocar-se, com segurança e com mais realismo.

O presente Termo de Consentimento Livre e Esclarecido será fornecido pelo pesquisador Daniel Stefany Duarte Caetano e deverá ser apresentado antes da realização do experimento no Núcleo de Tecnologias Assistivas da Universidade Federal de Uberlândia.

Após a conclusão de cada protocolo você deverá preencher um questionário parcial, descrevendo suas percepções sobre o funcionamento do sistema e do treinamento realizado. E após o último protocolo de treinamento, você deverá preencher também um questionário com perguntas pontuais sobre alguns requisitos do sistema, usabilidade e avaliação de sua experiência.

Em nenhum momento você será identificado, a não ser entre os responsáveis pelo estudo, sendo assegurado o sigilo sobre sua participação. Os resultados da pesquisa deverão ser publicados, porém ainda assim sua identidade será preservada.

Você não terá nenhum gasto e ganho financeiro por participar na pesquisa.

Os riscos envolvidos consistem na possível fadiga mental durante a realização dos protocolos de treinamento remoto.

Os benefícios que se espera com este estudo são: ter uma avaliação da arquitetura de sistema desenvolvida, para aprimorá-lo a partir das sugestões coletadas de cada participante. Futuramente, traçar-se-á abordagens de melhorias futuras, para que nas próximas etapas do projeto sejam implantadas, objetivando uma melhor a experiência no treinamento na condução de CRM além deixar a arquitetura mais robusta para toda a comunidade, permitindo alcançar melhores contribuições científicas com os dados futuramente obtidos.

Você é livre para deixar de participar da pesquisa a qualquer momento sem que haja qualquer prejuízo ou coação por parte dos envolvidos.

Uma cópia deste Termo de Consentimento Livre e Esclarecido ficará com você, e a segunda cópia será arquivada pelos pesquisadores.

Qualquer dúvida a respeito da pesquisa, você poderá entrar em contato com: Eduardo Lázaro Martins Naves (34) 3239-4769; Daniel Stefany Duarte Caetano (34) 99645-6207 (sdc.daniel@gmail.com) – Universidade Federal de Uberlândia: Av. João Naves de Ávila, nº 2121, bloco A, sala 220, Campus Santa Mônica – Uberlândia –MG. Poderá também entrar em contato com o Comitê de Ética na Pesquisa com Seres-Humanos – Universidade Federal de Uberlândia: Av. João Naves de Ávila, nº 2121, bloco A, sala 224, Campus Santa Mônica – Uberlândia –MG, CEP: 38408-100; fone: 34-32394131


\vspace{\baselineskip}

\vspace{\baselineskip}

\begin{center}
Uberlândia, 20 de Maio de 2019.
\end{center}

\vspace{\baselineskip}

\vspace{\baselineskip}

\vspace{\baselineskip}

\vspace{\baselineskip}

\begin{center}
\_\_\_\_\_\_\_\_\_\_\_\_\_\_\_\_\_\_\_\_\_\_\_\_\_\_\_\_\_\_\_\_\_\_\_\_\_\_\_\par

Assinatura do(s) pesquisador(es)\par
\end{center}


\vspace{\baselineskip}

\vspace{\baselineskip}
Eu aceito participar do projeto citado acima, voluntariamente, após ter sido devidamente esclarecido.

\begin{center}
\vspace{\baselineskip}
\_\_\_\_\_\_\_\_\_\_\_\_\_\_\_\_\_\_\_\_\_\_\_\_\_\_\_\_\_\_\_\_\_\_\_\_\_\_\_\par

Participante da pesquisa\par
\end{center}


\vspace{\baselineskip}
Eu,\_\_\_\_\_\_\_\_\_\_\_\_\_\_\_\_\_\_\_\_\_\_\_\_\_\_\_\_\_\_\_\_\_\_\_\_\_\_\_\_\_\_\_\_\_\_
responsável pelo \_\_\_\_\_\_\_\_\_\_\_\_\_\_\_\_\_\_\_\_\_\_\_\_\_\_\_\_\_\_\_\_\_\_\_\_\_\_\_\_\_\_ autorizo e responsabilizo a participação do mesmo no projeto descrito acima.\par


\vspace{\baselineskip}


