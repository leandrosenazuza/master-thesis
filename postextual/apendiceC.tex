\chapter{Appendix - Consent Form}
\label{sec:consentFormEN}

\begin{center}\textbf{FREE AND CLARIFIED CONSENT TERM}
\end{center}


\vspace{\baselineskip}
You are being invited to participate in the research entitled ``An Augmented Reality-based Telerehabilitation Architecture for Supporting the Training of Powered Wheelchair Users'', under the responsibility of researchers Eduardo Lázaro Martins, Edgard Afonso Lamounier Jr, Alexandre Cardoso and Daniel Stefany Duarte Caetano.

This research intends to provide the conditions for rehabilitation or to recover its independence in carrying out activities that, before mobility, are restricted. It is believed that through the developed augmented telerehabilitation environment, you can remotely, with the intervention and monitoring of the therapist, performing personalized tasks that meet your needs. Based on your daily challenges, safely and as close to reality as possible. You and the therapist will be able to interact with this environment, whose infrastructure was created based on your applications, and virtual objects that will assist you in this process.

A fusion of the Augmented Reality (AR) techniques with the advanced features of Telerehabilitation, allows increased responses received, instantly, from the control commands sent in the remote PW. Using AR techniques, the therapist will define or execute it for you, where users can use it as guides, and others will be avoided. The therapist still evaluates, takes notes and monitors your training. All for you to perform several maneuvers to develop your PW control skills. You will be instructed on how to use and request training on the system, and through an adapted joystick, connected to the computer, you will send the commands to control a PW remotely without having to move, safely and with more realism.

This Free and Informed Consent Term will be used by the researcher Daniel Stefany Duarte Caetano and will be released before the experiment is carried out at the Assistive Technologies Center of the Federal University of Uberlândia.

After completing each protocol, you must complete a partial questionnaire, describing your perceptions of the system's functioning and the training conducted. After the last training protocol, you can also complete a questionnaire with specific questions about some system requirements, use and evaluation of your experience.

At no time will you be identified, you will not be among those responsible for the study, you will be assured of confidential about your participation. The results of the research must be published, but your identity will still be preserved.

You will have no financial expense and gain to participate in research.

The risks involved are possible mental fatigue during remote training protocols.

The benefits that await with this study are: an assessment of the architecture of the developed system, to improve it based on the suggestions collected from each participant. In the future, to track future improvement approaches, for the next stages of the project, be implemented, aiming at a better training experience in conducting PW, in addition to leaving a more robust architecture for the whole community, allowing better performances with the data obtained in the future.

You are free to stop participating in the research at any time without prejudice or coercion by those involved.

A copy of this Free and Informed Consent Form is blocked with you, and a second copy will be filed by the researchers.

Any questions regarding the research, you can contact: Eduardo Lázaro Martins Naves (34) 3239-4769; Daniel Stefany Duarte Caetano (34) 99645-6207 (sdc.daniel@gmail.com) - Federal University of Uberlândia: Av. João Naves de Ávila, nº 2121, block A, room 220, Campus Santa Mônica - Uberlândia - MG. You can also contact the Human Research Ethics Committee - Federal University of Uberlândia: Av. João Naves de Ávila, nº 2121, block A, room 224, Campus Santa Mônica - Uberlândia –MG, CEP: 38408-100 ; phone: 34-32394131


\vspace{\baselineskip}

\vspace{\baselineskip}

\vspace{\baselineskip}

\vspace{\baselineskip}

\begin{center}
Uberlândia, May 20, 2019.
\end{center}

\vspace{\baselineskip}

\vspace{\baselineskip}


\vspace{\baselineskip}

\vspace{\baselineskip}

\vspace{\baselineskip}

\vspace{\baselineskip}

\vspace{\baselineskip}

\vspace{\baselineskip}

\begin{center} 
\_\_\_\_\_\_\_\_\_\_\_\_\_\_\_\_\_\_\_\_\_\_\_\_\_\_\_\_\_\_\_\_\_\_\_\_\_\_\_\par 
Researcher(s) Signature(s)\par 

\end{center} 

\vspace{\baselineskip} 

\vspace{\baselineskip} 

I accept to participate in the project mentioned above, voluntarily, after having been duly clarified. 

\begin{center} 

\vspace{\baselineskip} \_\_\_\_\_\_\_\_\_\_\_\_\_\_\_\_\_\_\_\_\_\_\_\_\_\_\_\_\_\_\_\_\_\_\_\_\_\_\_\par 
Survey participant \par 
\end{center} 

\vspace{\baselineskip} 

I,\_\_\_\_\_\_\_\_\_\_\_\_\_\_\_\_\_\_\_\_\_\_\_\_\_\_\_\_\_\_\_\_\_\_\_\_\_\_\_\_\_\_\_\_\_\_\_
responsible for \_\_\_\_\_\_\_\_\_\_\_\_\_\_\_\_\_\_\_\_\_\_\_\_ 
authorize and hold responsible for its participation in the project described above.