\newpage
\thispagestyle{empty}
\chapter*{Resumo}
\vspace{-35pt}
A identificação precisa de equipamentos em imagens desempenha um papel crucial em várias operações relacionadas às subestações de energia, por facilitar não apenas a manutenção, mas o monitoramento dessas instalações. Neste trabalho, são apresentados resultados preliminares da eficiência da YOLOv8, na detecção do reator de núcleo de ar em imagens obtidas por Veículos Aéreos Não Tripulados (VANTs). Foram aplicadas diferentes técnicas de otimização a fim de obter melhor eficiência na detecção. A partir desse estudo, será possível estender o uso da técnica para agilizar o processo de geração de Ambientes Virtuais de Treinamento de procedimentos em subestação de energia.
\section*{Palavras Chave}
Subestação de Energia; VANTs; YOLOv8; Otimização; Ambientes Virtuais de Treinamento
\newpage

