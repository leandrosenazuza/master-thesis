\newpage
\thispagestyle{empty}
\chapter*{Resumo}
\vspace{-35pt}

A identificação precisa de equipamentos em imagens desempenha um papel crucial em várias operações relacionadas às subestações de energia, facilitando não apenas a manutenção, mas também o monitoramento dessas instalações. Neste trabalho, apresentamos resultados da eficiência da YOLOv8 na detecção de equipamentos presentes em subestações de energia, a partir de imagens obtidas por Veículos Aéreos Não Tripulados (VANTs). Utilizamos diferentes técnicas de otimização para aprimorar a eficiência na detecção, visando alcançar resultados mais precisos e rápidos. Além disso, este estudo visa ir além do mero treinamento com as fotos capturadas, buscando identificar o melhor modelo treinado para criar um script capaz de selecionar, a partir de uma base de modelos de realidade virtual, os elementos necessários para a montagem de uma subestação de energia virtual. Dessa forma, almejamos não apenas melhorar os processos de manutenção e monitoramento das subestações de energia na realidade física, mas também agilizar e aprimorar a geração de Ambientes Virtuais de Treinamento para procedimentos relacionados a essas subestações. Com este avanço, esperamos não só otimizar o uso de tecnologia de detecção em subestações de energia, mas também contribuir significativamente para a criação de ambientes virtuais realistas e eficientes para treinamento em procedimentos relacionados à operação e manutenção dessas instalações.

\section*{Palavras Chave}
Subestação de Energia; VANTs; YOLOv8; Otimização; Ambientes Virtuais de Treinamento
\newpage

