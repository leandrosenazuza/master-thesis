\newpage
\thispagestyle{empty}
\chapter*{Resumo}
\vspace{-35pt}

O treinamento com Realidade Virtual (RV) tem ganhado destaque nos ambientes corporativos e industriais. A realização de manobras educativas em ambientes virtuais reduz o risco que colaboradores inexperientes enfrentam em situações hostis e de alta periculosidade, como as encontradas em subestações de energia. Contudo, a modelagem desses ambientes exige mão de obra especializada e consome inúmeras horas de desenvolvimento. Diante disso, esta dissertação propõe um protótipo para automatizar a criação desses Ambientes Virtuais, utilizando a YOLOv8, uma Rede Neural Convolucional (RNC). Este trabalho é pioneiro e foca na identificação e inserção de Reatores de Núcleo de Ar, equipamentos essenciais nas subestações de energia. Para tanto, foram utilizadas imagens capturadas por Veículos Aéreos Não Tripulados (VANTs) em duas subestações de energia, selecionando aquelas em que o equipamento estava presente. Após essa seleção, as imagens foram submetidas ao treinamento das versões da YOLOv5 à YOLOv8, com variações metódicas de parâmetros para identificar a configuração que proporcionasse a melhor precisão. Os resultados do treinamento indicaram que a YOLOv8, em comparação com suas versões anteriores, demonstrou superioridade em termos de precisão, especialmente ao utilizar um tamanho de \textit{batch} igual a 16 e o otimizador SGD. Com o modelo de reconhecimento treinado com essa configuração, foi desenvolvida uma aplicação integrada ao software Unity, capaz de receber fotos não treinadas, identificar a quantidade de reatores presentes na imagem e buscar seus objetos modelados correspondentes, inserindo-os na cena de RV.

\section*{Palavras Chave}
Ambientes Virtuais de Treinamento; Otimização; Subestação de Energia; VANTs; YOLOv8
\newpage


