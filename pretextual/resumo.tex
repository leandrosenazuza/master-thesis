\newpage
\thispagestyle{empty}
\chapter*{Resumo}
\vspace{-35pt}
Muitas pessoas em todo o mundo estão vivenciando uma diminuição de sua mobilidade como resultado de envelhecimento, acidentes e doenças degenerativas. Em muitos casos, uma cadeira de rodas motorizada (CRM) é uma ajuda alternativa. Atualmente, no Brasil, os pacientes podem receber uma CRM do Sistema Único de Saúde, seguindo os critérios de prescrição. No entanto, eles não têm um treinamento prévio apropriado para dirigir a CRM. Conseqüentemente, os usuários podem sofrer acidentes, pois um protocolo de treinamento personalizado não está disponível. Além disto, devido a limitações financeiras e / ou de saúde, muitos usuários não podem comparecer a um centro de reabilitação. Para superar essas limitações, desenvolvemos uma arquitetura de sistema de telereabilitação com Realidade Aumentada (RA) baseado no PMRT (Power Mobility Road Test), para apoiar o treinamento de usuários de CRM. Nesse sistema, os terapeutas podem personalizar e avaliar remotamente as tarefas de treinamento e o usuário pode realizar o treinamento em condições mais seguras. O fluxo de vídeo e a transferência de dados entre cada ambiente foram possíveis através do UDP (User Datagram Protocol). Para avaliar e apresentar o potencial da arquitetura do sistema, foi realizado um teste preliminar de três participantes com lesão medular. Eles realizaram três protocolos básicos de treinamento definidos por um terapeuta. As seguintes métricas adotadas para avaliação foram: número de comandos de controle; tempo decorrido; número de colisões e biossinais. Além disso, um questionário foi usado para avaliar os recursos do sistema. Os resultados demonstram as necessidades específicas dos indivíduos que usam uma CRM, graças às métricas adotadas (qualitativas e emocionais). Além disso, os resultados mostraram o potencial do sistema de treinamento com protocolos personalizáveis para atender a essas necessidades. A avaliação do usuário demonstra que a combinação de técnicas de RA com adaptações PMRT aumenta o bem-estar do usuário após as sessões de treinamento. Além disso, esta experiência de treinamento ajuda os usuários a superar seus problemas de deslocamento, bem como a apontar desafios antes do uso em larga escala. A arquitetura de sistema proposta, permite estudos adicionais sobre a telerreabilitação de usuários de CRM.
\section*{Palavras Chave}
 Treinamento em Cadeira de Rodas Motorizada, Realidade Aumentada, Telerreabilitação, Power Mobility Road Test, Biosinais.
\newpage

