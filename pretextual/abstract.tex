\newpage
\thispagestyle{empty}
\chapter*{Abstract}
\vspace{-35pt}

The precise identification of equipment in images plays a crucial role in various operations related to power substations, facilitating not only maintenance but also the monitoring of these facilities. In this work, we present results on the efficiency of YOLOv8 in detecting equipment present in power substations from images obtained by Unmanned Aerial Vehicles (UAVs). We employ different optimization techniques to enhance detection efficiency, aiming to achieve more accurate and faster results. Additionally, this study aims to go beyond mere training with captured photos, seeking to identify the best-trained model to create a script capable of selecting, from a database of virtual reality models, the elements necessary for assembling a virtual power substation. Thus, we aim not only to improve the maintenance and monitoring processes of power substations in physical reality but also to streamline and enhance the generation of Virtual Training Environments for procedures related to these substations. With this advancement, we hope to not only optimize the use of detection technology in power substations but also to significantly contribute to the creation of realistic and efficient virtual environments for training in procedures related to the operation and maintenance of these facilities.
\section*{Keywords}
Keywords - Power Substation; UAV; YOLOv8; Optimization; Virtual Training Environments
