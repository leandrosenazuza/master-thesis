\newpage
\thispagestyle{empty}
\chapter*{Abstract}
\vspace{-35pt}
Many people worldwide have been experimenting a decrease in their mobility as a result of aging, accidents and degenerative diseases. In many cases, a Powered Wheelchair (PW) is an alternative help. Currently, in Brazil, patients can receive a PW from the Unified Health System, following prescription criteria. However, they do not have an appropriate previous training for driving the PW. Consequently, users might suffer accidents since a customized training protocol is not available. Nevertheless, due to financial and/or health limitations, many users are unable to attend a rehabilitation center. To overcome these limitations, we developed an Augmented Reality (AR) Telerehabilitation System Architecture based on the Power Mobility Road Test (PMRT), for supporting PW user’s training. In this system, the therapists can remotely customize and evaluate training tasks and the user can perform the training in safer conditions. Video stream and data transfer between each environment were made possible through UDP (User Datagram Protocol). To evaluate and present the system architecture potential, a preliminary test was conducted with 3 spinal cord injury participants. They performed 3 basic training protocols defined by a therapist. The following metrics were adopted for evaluation: number of control commands; elapsed time; number of collisions; biosignals and a questionary was used to evaluate system features by participants. Results demonstrate the specific needs of individuals using a PW, thanks to adopted (qualitative and emotional) metrics.  Also, the results have shown the potential of the training system with customizable protocols to fulfill these needs. User’s evaluation demonstrates that the combination of AR techniques with PMRT adaptations, increases user’s well-being after training sessions. Furthermore, a training experience helps users to overcome their displacement problems, as well as for appointing challenges before large scale use. The proposed system architecture allows further studies on telerehabilitation of PW users.
\section*{Keywords}
Powered Wheelchair Training, Augmented Reality, Telerehabilitation, Power Mobility Road Test, Biosignals.
