\newpage
\thispagestyle{empty}
\chapter*{Abstract}
\vspace{-35pt}

Training with Virtual Reality (VR) has gained prominence in corporate and industrial environments. Conducting educational maneuvers in virtual settings reduces the risk that inexperienced employees face in hostile and high-risk situations, such as those found in power substations. However, modeling these environments requires specialized labor and consumes countless hours of development. In light of this, this dissertation proposes a prototype to automate the creation of these Virtual Environments using YOLOv8, a Convolutional Neural Network (CNN). This work is pioneering and focuses on the identification and insertion of Air Core Reactors, essential equipment in power substations. To achieve this, images captured by Unmanned Aerial Vehicles (UAVs) in two power substations were used, selecting those in which the equipment was present. After this selection, the images were subjected to training using versions from YOLOv5 to YOLOv8, with methodical variations of parameters to identify the configuration that provided the best accuracy. The training results indicated that YOLOv8, compared to its previous versions, demonstrated superiority in terms of accuracy, particularly when using a batch size of 16 and the SGD optimizer. With the recognition model trained with this configuration, an application integrated into Unity software was developed, capable of receiving untrained photos, identifying the number of reactors present in the image, and retrieving their corresponding modeled objects, inserting them into the VR scene.

\section*{Keywords}
Virtual Training Environments; Optimization; Power Substation; UAV; YOLOv8