\newpage
\thispagestyle{empty}
\chapter*{Acknowledgements}

We often believe that we develop a job for someone, to benefit someone! However, in the middle of the process, we are feeling that we are the ones being worked. About how to look, how to develop our feelings and sensibility, in the strengthening of ties, which makes us grow as human beings. I thank of my entire self, for having had the opportunity to be with each one of you, volunteers, professionals, who enriched me and made it possible, to better understand one of these facets of the silent pain experienced by the powered wheelchair users.

I thank God for nurturing me in many ways.

I also thank those that came before me and continue with me, through the eternal link of Love, and that gave me support to be here.

Thanks in particular to my advisor and co-advisor Dr. Edgard Lamounier and Dr. Eduardo Naves for their masterful guidance, encouragement and support.

To my colleagues and lab friends for sharing knowledge and support in daily activities.

To professors and professionals who were in this journey.

To the examination board who agreed to participate and collaborate in the evaluation
and enrichment of this work.

To my wife, Lorraine Caroline, for her patience, support and encouragement. Her
contribution was essential.

To my parents and siblings for their love and support in many situations.

To my friends, with whom I have shared various moments of my life, they are responsible for much of the brightness of each day.

The present work was carried out with the support of the Coordination for the Improvement of
Higher Education Personnel (CAPES) under the project identification CAPES PGPTA 3461/2014. I am thankful to the volunteers and to the patients come from Disabled Child Care Association of Uberlândia (Associação de Assistência à Criança Deficiente-AACD), Association of Paraplegics of Uberlândia (Associação dos Paraplégicos de Uberlândia-APARU) for their participation in this study.

