\chapter{Conclusão e Trabalhos Futuros}

\section{Introdução} 

Neste trabalho, foi proposto um sistema de inserção automática de um equipamento específico de uma subestação de energia em um software de geração de ambientes de RV. Para isso, concorreram a captura de imagens aéreas para ser substrato ao treinamento da RNC; recorreu-se a uma base de modelos digitalizados do equipamento estudado; e foi desenvolvido um sistema automático que vincule essa peça em um ambiente virtual, apenas fornecendo uma imagem em que a mesma esteja representada em uma visão aérea. Esta seção apresentará uma conclusão sobre todo o estudo desenvolvido, assim como trabalhos futuros que poderão partir desse, tanto como melhorias, assim como outras contribuições possíveis à comunidade científica, no devido contexto.

\section{Conclusão} 

Neste estudo, foi desenvolvido um sistema inovador que permite a inserção automática de equipamentos específicos de subestações de energia em ambientes de RV. Por meio da captura de imagens aéreas e do treinamento da RNC YOLOv8, conseguiu-se identificar e posicionar automaticamente os equipamentos em um ambiente virtual. Além disso, a implementação de uma automação com o software Unity permitiu integrar de forma eficiente os modelos digitalizados dos equipamentos ao ambiente de RV, garantindo uma representação consideravelmente precisa, assim como intuitiva.

Além disso, com esta dissertação foi possível validar a superioridade de precisão no uso da YOLOv8 para o conjunto de imagens coletadas em subestações de energia por meio de VANTs, em relação a suas últimas versões. Isso confirma a extrapolação feita a partir do trabalho de \cite{ultralytics2023yolo}, para o conjunto de dados aqui apresentado. 

Também neste trabalho, conclui-se que o estudo apresentado por \cite{gonzaga2023identificaccao} mostrou-se substrato essencial para chegar à melhor configuração dos parâmetros da YOLO para obter a melhor precisão em sua oitava versão.

Os resultados abrem espaço para uma grande variedade de trabalhos que busquem utilizar de imagens aéreas combinadas com RNC, aplicadas à construção de ambientes de RV. Essa abordagem, se explorada ainda mais a fundo para outros equipamentos presentes em toda extensão da subestação, poderá elevar o nível da automação da criação de ambientes virtuais, poupando esforço manual e, com os devidos ajustes, aumentando a precisão da criação.

\section{Trabalhos Futuros} 

Este trabalho propõe-se como o ponto de partida para uma série de futuras implementações e estudos. Primeiramente, a precisão e a eficiência do sistema podem ser aprimoradas com o uso de datasets mais extensos e variados para o treinamento da YOLOv8. 

Naturalmente, uma melhora considerável na aplicabilidade desse sistema consistiria na implementação na \textit{API}, para que haja um esforço de geolocalização de cada equipamento, a fim de que, além da detecção da quantidade de objetos presentes na foto, seja também posicionado corretamente cada um deles, conforme sua configuração real na subestação de energia. 

Finalmente, espera-se que este trabalho sirva como base para novas pesquisas na área de realidade virtual e inteligência artificial aplicada, contribuindo para o avanço tecnológico e oferecendo novas ferramentas e metodologias para a comunidade científica e industrial.


