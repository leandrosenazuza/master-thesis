\chapter{Conclusão e Trabalhos Futuros}

\section{Introdução} 

Neste trabalho, foi proposto um sistema de inserção automática de um equipamento específico de uma subestação de energia em um software de geração de ambientes de RV. Para isso, concorreram a captura de imagens aéreas para ser substrato ao treinamento da RNC; recorreu-se a uma base de modelos digitalizados do equipamento estudado; e foi desenvolvido um sistema automático que vincule essa peça em um ambiente virtual, apenas fornecendo uma imagem em que a mesma esteja representada em uma visão aérea. Esta seção apresentará uma conclusão sobre todo o estudo desenvolvido, assim como trabalhos futuros que poderão partir desse, tanto como melhorias, assim como outras contribuições possíveis à comunidade científica, no devido contexto.

\section{Conclusão} 

Neste estudo, foi desenvolvido um sistema inovador que permite a inserção automática de equipamentos específicos de subestações de energia em ambientes de RV. Através da captura de imagens aéreas e do treinamento da RNC YOLOv8, conseguimos identificar e posicionar automaticamente os equipamentos em um ambiente virtual. Além disso, a implementação de uma automação com o software Unity permitiu integrar de forma eficiente os modelos digitalizados dos equipamentos ao ambiente de RV, garantindo uma representação precisa e interativa.

Os resultados demonstraram que há uma forte correção entre imagens aéreas e a possibilidade de identificar adequadamente os equipamentos. Essa abordagem, se explorada ainda mais a fundo para outros equipamentos presentes em toda extensão da subestação, poderá elevar o nível da automação da criação de ambientes de RV, poupando esforço manual e, com os devidos ajustes, aumentando a precisão da criação.

\section{Trabalhos Futuros} 

Este trabalho propõe-se como o ponto de partida para uma série de futuras implementações e estudos. Primeiramente, a precisão e a eficiência do sistema podem ser aprimoradas com o uso de datasets mais extensos e variados para o treinamento da YOLOv8. Além disso, a integração com outras ferramentas de software e hardwares, como drones com câmeras de maior resolução e sistemas de RV mais avançados, pode potencializar ainda mais a aplicabilidade e a funcionalidade do sistema.

Outra área de desenvolvimento futuro envolve a adaptação do sistema para outros tipos de equipamentos e contextos industriais, ampliando seu uso para diferentes setores. O refinamento do sistema de automação com Unity também pode incluir a adição de funcionalidades mais avançadas, como simulações em tempo real e interatividade aprimorada.

Finalmente, espera-se que este trabalho sirva como base para novas pesquisas na área de realidade virtual e inteligência artificial aplicada, contribuindo para o avanço tecnológico e oferecendo novas ferramentas e metodologias para a comunidade científica e industrial.
