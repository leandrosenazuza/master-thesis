\chapter{Conclusão e Trabalhos Futuros}


\section{Introdução} 

Neste trabalho, foi proposto um sistema de inserção automática de um equipamento específico de uma subestação de energia em um software de geração de ambientes de RV. Para isso, concorreram a captura de imagens aéreas para ser substrato ao treinamento da RNC; recorreu-se a uma base de modelos digitalizados do equipamento estudado; e foi desenvolvida um sistema automático que vincule essa peça em um ambiente virtual, apenas fornecendo uma imagem em que a mesma esteja representada em uma visão aérea. Esta seção apresentará uma conclusão sobre todo o estudo desenvolvido, assim como trabalhos futuros que poderão partir desse, tanto como melhorias, assim como outra contribuições possíveis à comunidade científica, no devido contexto.

\section{Conclusão} 

Conclusão

\section{Trabalhos Futuros} 

Antes de tudo, este trabalho propõe-se como o ponto de partida para uma série de futuras implementações e estudos. Aqui, foram treinadas imagens para  




%\chapter{Conclusão e Trabalhos Futuros}
%
%
%\section{Introdução} 

%In this work, an augmented telerehabilitation architecture for supporting PW driving skills training is proposed. Being aware of PW user's abilities and limitations in the rehabilitation process (get back to perform daily activities), driving skills might be developed. Also, the therapist responsible for the rehabilitation process,  finds necessary to identify resources to meet the individual user needs, creating and evaluating training. For this, telerehabilitation and AR techniques are presented as a solution. Thus, it is hoped that, in the future, travel restrictions faced by users and so that the lack of resources (physical and technological) for the therapist will no longer be a barrier as well.  This section presents the conclusions about the study carried out and also future work that can be presented as an improvement or as possible contributions to the scientific community within the context presented. 

%\section{Conclusão} 

%Not only in Brazil the PW is considered the most recommended AT for people who, for various reasons, have a mobility decrease, preventing them from carrying out ADL normally \cite{censo2010, john2018, caro2018}. There are many users around the world who depend on it to regain their independence, joy and good mental and emotional health \cite{john2018, caro2018, macgillivray2018, censo2010}. However, if good training is not carried out, in addition to AT being abandoned, the user might injury and lose their life quality \cite{who2008, caro2018, macgillivray2018,dorrington2016, pettersson2014}. As demonstrated in this work, there are several limitations and therapists encounter difficulties to offer an appropriate care. These are some reasons for the development of the proposed solution, to be support for the reintegration of the user's life.
%
%The questionnaires were applied before the beginning of the research development so that the solution was a user-centered design \cite{dorrington2016}. Besides, as the therapist is an essential part of the process, professional considerations and literature have been consulted \cite{valentini2019}. Then, it was detected that users needed personalized training to reduce wear and also meet their daily needs and still be carried out safely. And also, a therapist needs a tool that allows him to personalize, evaluate and follow the user's training.
%
%During the development process, three distinct sites, physically separated, were defined: training, therapist and patient. The combination of WebRTC and AR techniques enables Telerehabilitation architecture development. The remotely accessed training site was fully developed based on user's surveys. It has a PW properly equipped to be remotely controlled and provides the visualization of remote control requests. The therapist site, through which the therapist remotely creates, defines, tracks (from an augmented environment preview) and evaluates user training. At last, the patient's site is the environment in which the user will carry out his training session. In this environment, a conventional joystick was adapted, to be connected to the computer where it will interact (insert commands and having an augmented environment preview) with the training site.
%
%Thus, based on the characteristics of the architecture and the training experience that the user will have, a questionnaire was developed. The objective is to evaluate the quality of the experience that he had, in addition to verifying whether the developed resources allow to reach the initials objective set. Then, based participants' survey answers, we believe that the purpose of the system architecture was achieved with some limitations. In this preliminary study, all tests were conducted in a rehabilitation center. 
%
%From the results of these experiments, in general, the architecture met its objectives, because 77\% of the questions were evaluated above the average score. Questions (Q2, Q6 and Q9) obtained 100\% of users' satisfaction and among the items evaluated are: the graphical interface, virtual objects and well-being after training. Other questions (Q1, Q5, Q8, Q10 and Q11) did not obtain 100\% satisfaction, which refers to improvements related to the use of the system, image quality, the ability of the tool to assist in the development of driving skills, system characteristics satisfaction and whether the system does what has been proposed. Much of this assessment is due to the difficulty in controlling the PW, due to latency and lack of diagonal movements, which interfere with navigation experience making it difficult to use the system at the training time. All of this is related to the low evaluation in the questions (Q3, Q4 and Q7).
%
%It can also be observed that due to the adaptation at the moment of composing the activities of each protocol, they increased the users' well-being after training, which is important in terms of performance. Another observation is that the defined evaluation methodology allows the therapist, based on the visual perception of the training, to evaluate the user with quality and also monitor their development through comparative graphics. Another contribution of this work is related to collected biological signals. It allows, from an emotional point of view, to check other variables that may be interfering with the user's performance. From these results, it is concluded that the protocols did not show the difference in the participants. Participant 3 was the only one to change the stress zones. However, it is believed that it represents a normal variation towards a state of rest in the normal stress zone, not caused by the protocols.

%2\section{Trabalhos Futuros} 
%2
%2    Antes de tudo, este trabalho propõe-se como o ponto de partida para uma série de futuras implementações e estudos. Aqui, foram treinadas imagens para  
%Future work can be divided into two strands: increase users’ experience and scientific contributions. Based on the ADL driving challenges survey to enrich the next training site versions with different assets not contemplated, for example, holes. In order to correct the rendering failure of the augmented objects, fiducial markers shall be removed and replaced by rendering based on the Cartesian position of the PW \cite{raul2016}. By implementing a proximity sensor, all virtual objects within range are rendered, as in the Poke-mon GO game \cite{el2019}. Otherwise, the use of deep learning techniques can be investigated in order to reduce rendering failure \cite{mattioli2019}.  Studies will be conducted, to improve the control signals and video stream latency. With the release of 5G Networks, many problems related to latency faced in this work due to the 4G Internet and Internet speed provide by concessionary, in the rehabilitation center can be addressed \cite{rashid2019}. Finally, collect the therapist's opinion about the system architecture, to replace the PW model used by another one with less inertia breaking time and also with diagonal movements, for increasing user navigation meets and system usage.
%
%Based on the volunteer evaluations and the discussions carried out, pointed out that the video stream latency is a major limiter in the user's experience quality. Several times, it was not possible to start the training process because very slow Internet speed, so that the video connection between the environments was not completed, due to the framework quality restrictions. For this reason, it was decided to interrupt the collections, not going forward with the other existing volunteers. Since the maximum time previously agreed with each one, to perform the experiment, was exceeded. With this, the volunteers could get tired, due to the individual restrictions shown by Table \ref{table:tbUserClinical}, or concerned with other activities to be developed, or even the taxi used for commuting, could not wait any longer. For this reason, was carried out with only one collection of each of the three volunteers.
%
%
%However, in the future, after the issues found in this research have been fixed, only the first session will be performed in a rehabilitation center and futures sessions might be accomplished at home, in agreement with the user conditions. In addition, different subject segmentation based on the type of injury, severity, age and experience in the use of assistive technologies can be implemented to assess the learning curve connected to the evolution of the participants' biosignals over time and identify obstacles in each protocol. Monitoring user stress is also important not only to address the learning but also to ensure safety during regular use. Also, further studies should be done in this direction to classify the user's stress better and provide a risk assessment to develop improved strategies to prevent accidents.
%
%Since the beginning of the system architecture development, it was opted to use only open-source tools and libraries. It is hoped in the future to incorporate this system on the rehabilitation centers connected with Brazilian Unified Health System and share\footnote{https://github.com/dantutu/servletserver} this project with the community as open-source for non-commercial license use.

 