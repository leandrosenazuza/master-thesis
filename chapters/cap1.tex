\chapter{Introdução}

\section{Considerações Preliminares}

As subestações de energia desempenham um papel fundamental no sistema de distribuição de energia elétrica de todo o país, permitindo a transferência eficiente e segura de eletricidade entre diferentes níveis de tensão. Elas são cruciais para garantir que a eletricidade gerada em usinas seja entregue aos consumidores com a qualidade e confiabilidade necessárias. Além disso, as subestações desempenham um papel crucial na estabilidade e na segurança do sistema elétrico, facilitando a manutenção, controle e proteção da rede. Seu funcionamento envolve várias etapas, começando com a recepção da eletricidade gerada em usinas de energia. A eletricidade é então transformada em níveis de tensão adequados para distribuição por meio de transformadores. Nas subestações, também ocorrem operações de chaveamento, onde os dispositivos de comutação são usados para controlar o fluxo de eletricidade e direcioná-lo para as áreas desejadas da rede. Além disso, as subestações estão equipadas com sistemas de proteção que detectam e isolam falhas para evitar danos ao sistema elétrico e garantir a segurança dos equipamentos e dos operadores \cite{randolph2013electric}.

\section{Motivação}
\label{sec:motivation}

Nesse sentido, intervenções tecnológicas inovadoras desempenham um papel crucial na melhoria das operações das subestações de energia. Em um ambiente onde eficiência e segurança são fundamentais, tecnologias que agregam valor mostram-se muito bem-vindas. Desde sistemas avançados de monitoramento até soluções de Realidade Virtual, oferecendo oportunidades para aprimorar a gestão e operação das subestações. Ao incorporar essas inovações, as empresas de energia podem aumentar a resiliência do sistema elétrico, reduzir custos operacionais e garantir um fornecimento de energia mais confiável para os consumidores \cite{zhou2016big}.

\section{Objetivos e Metas}

The objective of this research is to evaluate the viability of a telerehabilitation environment with augmented reality techniques for the training of PW driving skills. To achieve this, the following goals were defined:
\begin{itemize}
\item To condute a literature review;
\item To elucidate the main components of computer assisted rehabilitation systems, applied to PW training;
\item To propose the main components to be used in each environment;
\item To develop a web-server responsible for handling, measuring and saving information generated within the environments;
\item To perform tests to assess the main application;
\item To evaluate the solution with potential users; and
\item To evaluate qualitative and emotional PW users training experience.
\end{itemize}

\section{Thesis organization}

A presente dissertação é composta por sete capítulos, descritos da seguinte forma.
\begin{itemize}
\item O Capítulo 1 descreve a motivação, objetivo e objetivos deste trabalho;
\item O Capítulo 2 introduz o contexto dos conceitos utilizados na solução apresentada;
\item O Capítulo 3 apresenta trabalhos relacionados;
\item Nos Capítulos 4 e 5 são apresentados materiais/métodos e detalhes de implementação;
\item No Capítulo 6, são discutidos os resultados preliminares obtidos nesta pesquisa e;
\item Finalmente, o Capítulo 7 apresenta as conclusões e desenvolvimentos futuros desta pesquisa.
\end{itemize}