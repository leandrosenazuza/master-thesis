\chapter{Introduction}


\section{Motivation}
\label{sec:motivation}

In Brazil, more than 45 million people have some motor limitation. Among these, 2.33 percent (1 million people) have a severe motor disability, according to the 2010 census \cite{gonzaga2023}. Mostly are elderly people whose autonomy is seriously affected by a decline in their motor function and cognitive performance. Also, it includes individuals who have suffered a stroke or injury \cite{gonzaga2023}. According to the Census of England and Wales, carried out in 2011, 1.9 percent of the population use a wheelchair, an estimated 1.2 million people \cite{gonzaga2023}. Other countries will have, proportionally, similar numbers within their population.



\section{Objectives and Goals}

The objective of this research is to evaluate the viability of a telerehabilitation environment with augmented reality techniques for the training of PW driving skills. To achieve this, the following goals were defined:
\begin{itemize}
\item To condute a literature review;
\item To elucidate the main components of computer assisted rehabilitation systems, applied to PW training;
\item To propose the main components to be used in each environment;
\item To develop a web-server responsible for handling, measuring and saving information generated within the environments;
\item To perform tests to assess the main application;
\item To evaluate the solution with potential users; and
\item To evaluate qualitative and emotional PW users training experience.
\end{itemize}

\section{Thesis organization}

The present thesis is composed of seven chapters, described as follows.
\begin{itemize}
\item Chapter 1 describes the motivation, aim, and objectives of this work;
\item Chapter 2 introduces the background of the concepts used in the
Presented solution;
\item Chapter 3 presents related work;
\item In Chapter 4 and 5 present, materials/methods, and implementation details;
\item In Chapter 6, the preliminary results obtained in this research are discussed and;
\item Finally, Chapter 7 presents the conclusions and future developments of this research.
\end{itemize}