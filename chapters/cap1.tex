\chapter{Introdução}

\section{Motivação}
\label{sec:motivation}

As subestações de energia desempenham um papel fundamental no sistema de distribuição de energia elétrica no Brasil, permitindo a transferência eficiente e segura de eletricidade entre diferentes níveis de tensão. Elas são cruciais para garantir que a eletricidade gerada em usinas seja entregue aos consumidores com a qualidade e confiabilidade necessárias. Efetivamente, desempenham um papel crucial na estabilidade do sistema elétrico, facilitando a manutenção, controle e proteção da rede. Seu funcionamento envolve várias etapas, começando com a recepção da eletricidade gerada em usinas de energia. A eletricidade é então transformada em níveis de tensão adequados para distribuição por meio de transformadores. Nas subestações, também ocorrem operações de chaveamento, onde os dispositivos de comutação são usados para controlar o fluxo de eletricidade e direcioná-lo para as áreas desejadas da rede. As subestações também estão equipadas com sistemas de proteção que detectam e isolam falhas para evitar danos ao sistema elétrico e garantir a segurança dos equipamentos e dos operadores \cite{randolph2013electric}.

Contudo falhas de segurança durante as rotinas de um colaborador não são raras no setor elétrico, podendo causar danos a sua saúde, e em alguns casos levando a óbito. De acordo com o estudo de \cite {lima2021precarizaccao}, acidentes de trabalho são muito comuns em ambientes como subestações de energia. Para avaliar as causas, foram analisados diversos processos trabalhistas durante um período de tempo contra empresas que prestam serviço neste setor. Foram concluídas as existência de várias as causas, mas uma se destaca: a falta de treinamento. No estudo citado, este fato é atrelado à terceirização dos serviços. Enquanto um funcionário direto da companhia de energia da região recebia 6 meses de treinamento, o funcionário terceirizado era treinado em um período médio de 30 a 40 dias. A qualidade do conteúdo destes treinamentos para o funcionário terceirizado também era muito mais superficial. Em uma análise dentro deste estudo, a respeito de um acidente fatal, foi verificado que um técnico foi acionado para resolução de um problema de rompimento de cabo. Devido a um descuido, um dos colaboradores tocou em um cabo energizado, sem se preocupar em verificar se todas chaves estariam desligadas, acabando por o levar a óbito. Deste estudo, portanto, entendeu-se que a busca por custos mais baixos durante o treinamento, pessoas que se expõe ao risco, e o próprio funcionamento da transmissão de energia é colocado à prova. 

Nesse cenário, intervenções tecnológicas seriam de grande valia para melhorar a condição de treinamento de operadores no sistema elétrico. São diversas as possibilidades de aplicações que podem atuar nesse sentido, desde sistemas avançados de monitoramento até soluções de Realidade Virtual (RV), oferecendo oportunidades para aprimorar a gestão e operação das subestações. A incorporação dessas inovações, podem levar as empresas do setor elétrico a oferecer maior segurança aos seus colaboradores, reduzir custos operacionais e garantir um fornecimento de energia mais confiável para os consumidores finais  \cite{zhou2016big}.

Para funções didáticas, como treinamentos, a RV se destaca como uma abordagem disruptiva em relação a métodos tradicionais, principalmente pelo alto nível de imersibilidade no contexto da aplicação, proporcionado ao usuário e ao alto resultado no aprendizado do conteúdo trabalhado. Contudo, para construir e preparar todo o ambiente para uma experiência imersiva em RV, faz-se necessário a elaboração de uma complexa estrutura que envolve desde a escolha do equipamento que será utilizado para a projeção ao usuário, como por exemplo, uma caverna de visualização, capacete de virtualização ou mesmo óculos de RV, até a criação, em softwares próprios para esse tipo de desenvolvimento, toda modelagem gráfica do ambiente até as interações que existentes na aplicação. Fatores como a capacidade gráfica e técnica são levadas em consideração nesta etapa, uma vez que aplicações com grande quantidade de interações e elementos, exigem do hardware que irá renderizar elevada capacidade de processamento. Se a demanda pela capacidade for alta, e não for suportada pelo hardware, será exigido do desenvolvedor redução na qualidade das texturas, assim como outros tratamentos para que toda a experiência durante a imersão não seja lenta ou mesmo careça de elementos que destitua a aplicação de imersibilidade \cite {palmeira2020uncanny}. 

Outro recurso que tem sido aplicado em várias áreas da ciência são as Redes Neurais Artificiais (RNA). Sua utilização tem sido atrelado a resolução de sistemas não-lineares em que nem todas as variáveis do problema são conhecidas, assim como problemas em que exista ruídos nos dados a serem tratados, ou seja, ideais para problemas do mundo real quando transportados para o mundo virtual. Ao simular o funcionamento do cérebro humano, replicando o aprendizado natural, as RNA exibem a capacidade de resolver problemas complexos, sendo, assim, ferramenta ótima a ser associada a um trabalho de pesquisa  \cite {ougcu2012forecasting}.

Deste modo, motivado pelo possibilidade de elaborar uma ferramenta que simplifique o desenvolvimento de um sistema em RV voltado para aplicações de treinamento de colaboradores em subestações de energia, este trabalho se propõe a construir uma ferramenta que faça a inserção automática de componentes em um ambiente de RV de uma subestação de energia. Toda a automação será alimentada por um modelo treinado a partir de uma RNA, alimentada por fotos capturadas por VANTs em duas subestações de energia diferentes.

\section{Objetivos e Metas}

O objetivo geral desta pesquisa é propor um sistema de inserção automática de componentes em ambientes virtuais de treinamento para subestações de energia a partir  de imagens aéreas coletadas do local a ser mapeado virtualmente. Para alcançar esse objetivo geral, foram estabelecidos os seguintes objetivos específicos:

\begin{itemize}
\item Realizar uma revisão da literatura científica, para identificar quais os algoritmos utilizados no reconhecimento de padrões em imagens aéreas obtidas por VANTs;
\item Estudar e avaliar quais são os hiperparâmetros do algoritmo de inteligência artificial a ser utilizado, que garanta maior eficiência no reconhecimento de componentes das subestações elétricas;
\item Desenvolver uma automação capaz de receber uma imagem, reconhecer componente(es) da subestação elétrica inserir no ambiente virtual.
\end{itemize}

\section{Estrutura da Dissertação}

A presente dissertação é composta por sete capítulos, descritos da seguinte forma.
\begin{itemize}
\item No Capítulo 1 são aparesentadas as motivações, os objetivos e a estruturação do trabalho;
\item No Capítulo 2 são aparesentados os principais fundamentos teóricos relacionados ao trabalho;
\item No Capítulo 3 é apresentado o estado da arte da linha de pesquisa principal desse trabalho;
\item Nos Capítulos 4 e 5 são apresentados materiais/métodos e detalhes de implementação;
\item No Capítulo 6, são discutidos e apresentados os resultados obtidos nesse trabalho a partir do sistema desenvolvido;
\item Por fim, no Capítulo 7, são aparesentadas as conclusões e as perspectivas para trabalhos futuros.
\end{itemize}