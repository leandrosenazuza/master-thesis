\chapter{Introdução}

\section{Considerações Preliminares}

As subestações de energia desempenham um papel fundamental no sistema de distribuição de energia elétrica de todo o país, permitindo a transferência eficiente e segura de eletricidade entre diferentes níveis de tensão. Elas são cruciais para garantir que a eletricidade gerada em usinas seja entregue aos consumidores com a qualidade e confiabilidade necessárias. Além disso, as subestações desempenham um papel crucial na estabilidade e na segurança do sistema elétrico, facilitando a manutenção, controle e proteção da rede. Seu funcionamento envolve várias etapas, começando com a recepção da eletricidade gerada em usinas de energia. A eletricidade é então transformada em níveis de tensão adequados para distribuição por meio de transformadores. Nas subestações, também ocorrem operações de chaveamento, onde os dispositivos de comutação são usados para controlar o fluxo de eletricidade e direcioná-lo para as áreas desejadas da rede. Além disso, as subestações estão equipadas com sistemas de proteção que detectam e isolam falhas para evitar danos ao sistema elétrico e garantir a segurança dos equipamentos e dos operadores \cite{randolph2013electric}.

Nesse sentido, intervenções tecnológicas inovadoras desempenham um papel crucial na melhoria das operações das subestações de energia. Em um ambiente onde eficiência e segurança são fundamentais, tecnologias que agregam valor mostram-se muito bem-vindas. Desde sistemas avançados de monitoramento até soluções de Realidade Virtual, oferecendo oportunidades para aprimorar a gestão e operação das subestações. Ao incorporar essas inovações, as empresas de energia podem aumentar a resiliência do sistema elétrico, reduzir custos operacionais e garantir um fornecimento de energia mais confiável para os consumidores \cite{zhou2016big}.

\section{Motivação}
\label{sec:motivation}


\section{Objetivos e Metas}

O objetivo dessa pesquisa é propor um sistema de construção automática de modelos de realidade virtual, baseado em redes neurais, de subestações de energia a partir  de imagens áreas coletadas do local a ser mapeado virtualmente. Para alcançar esse objetivo geral, pretende-se alcançar os seguintes objetivos específicos:
\begin{itemize}
\item Realizar uma revisão da literatura científica, justificando qual arquitetura de rede neural mais adequada para ser aplicada em imagens obtidas por Drone.
\item Avaliar a melhor configuração de parâmetros da rede neural YOLOv8, a fim de obter a maneira mais eficiente de se treinar todo dataset de fotos coletadas de subestações de energia, para geração de pesos para identificação dos equipamentos a partir de imagens;
\item Elaborar uma aplicação que utilize os pesos treinados para, a partir de uma imagem, verificar quais são os equipamentos presentes na imagem fornecida, e realizar uma busca em uma base de modelos modelos virtuais, retornando todos componentes identificados já plotados no software de modelagem tridimensional Blender;
\item Determinar por meio de análises qualitativas o melhor arranjo para esta combinação.
\end{itemize}

\section{Estrutura da Dissertação}

A presente dissertação é composta por sete capítulos, descritos da seguinte forma.
\begin{itemize}
\item No Capítulo 1 são aparesentadas as motivações, os objetivos e a estruturação do trabalho;
\item No Capítulo 2 são aparesentados os principais fundamentos teóricos relacionados ao trabalho;
\item No Capítulo 3 é apresentado o estado da arte da linha de pesquisa principal desse trabalho;
\item Nos Capítulos 4 e 5 são apresentados materiais/métodos e detalhes de implementação;
\item No Capítulo 6, são discutidos e apresentados os resultados obtidos nesse trabalho a partir do sistema desenvolvido;
\item Por fim, no Capítulo 7, são aparesentadas as conclusões e as perspectivas para trabalhos futuros.
\end{itemize}